\documentclass[12pt]{article}
\usepackage{amsfonts}
\usepackage{amsmath}
\usepackage{bm}

\title{From global to regional effects: a comparison of the different approaches}

\newcommand{\Rd}{\mathbb{R}^d}
\newcommand{\xb}{\mathbf{x}}
\newcommand{\xc}{\mathbf{x_c}}
\newcommand{\fxc}{f^{(\xc)}}
\newcommand{\fxs}{f^{(x_s)}}
\newcommand{\Xcal}{\mathcal{X}}
\newcommand{\Ycal}{\mathcal{Y}}

\author{Vasilis Gkolemis}
\begin{document}
\maketitle
\section{Introduction}

\section{Background}

Let \(\Xcal \in \Rd\) be the \(d\)-dimensional feature space, \(\Ycal\) the target space and \(f(\cdot) : \Xcal \rightarrow \Ycal\) the black-box function.  We use index \(s \in \{1, \ldots, d\}\) for the feature of interest and \(c = \{1, \ldots, d\} - s\) for the rest.  For convenience, to denote the input vector, we use \((x_s, \xc)\) instead
of \((x_1, \cdots , x_s, \cdots, x_D)\) and, for random variables,
\((X_s, X_c)\) instead of \((X_1, \cdots , X_s, \cdots, X_D)\).  The
training set \(\mathcal{D} = \{(\xb^i, y^i)\}_{i=1}^N\) is sampled
i.i.d.\ from the distribution \(\mathbb{P}_{X,Y}\).  Finally,
\(f^{\mathtt{<method>}}(x_s)\) denotes how \(\mathtt{<method>}\)
defines the feature effect and \(\hat{f}^{\mathtt{<method>}}(x_s)\)
how it estimates it from the training set.

\section{Feature Effect}

The purpose of any feature effect (FE) method is to explain the
`black-box` function $f: \mathbb{R}^D \rightarrow \mathbb{R}$
using a Generalized Additive Model $f_{\mathtt{<method>}}(x) = c + f_1(x_1) + \cdots + f_D(x_D)$, as a global surrogate.

\subsection{Approaches}

\begin{table}[htbp]
    \centering
    \caption{Table Caption}
    \begin{tabular}{c|c|c}
      \hline
      \textbf{Name} & \textbf{Definition} \((f)\) & \textbf{Approximation} \((\hat{f})\)\\
      \hline
      \textbf{PDP} & $\mathbb{E}_{X_c}[f(x_s,X_c)]$ & $\frac{1}{N} \sum f(x_s,x_c^{(i)})$ \\
      \textbf{dPDP} & $\mathbb{E}_{X_c}[ \frac{\partial f(x_s,X_c)}{\partial x_s}]$ & $\frac{1}{N} \sum \frac{\partial f(x_s,x_c^{(i)})}{\partial x_s}$ \\
    \end{tabular}
\end{table}

    \section{Interaction Index}
    \subsection{Approaches}
    \section{Regional Effects}
    \subsection{Approaches}

    \section{Can we evaluate the approaches?}

    \subsection{Idea 1}

    We may split every $f: \mathbb{R}^D \rightarrow \mathbb{R}$ into
    a model without interaction between $\xc$ and $x_s$,
    i.e., $f_{ni}(\xb) = \fxs(x_s) +  \fxc(\xc)$,
    and the interaction term $\kappa(\xc, x_s)$:

    \[
      f(\xb) = \underbrace{\fxs(x_s) +  \fxc(\xc)}_{f_{ni}(\xb)} + \kappa(\xc, x_s)
    \]

    A simple approach is defining \(f\) to be a Neural Network and \(f_{ni}\) a Neural Additive Model without interaction
    between \(x_s\) and \(\xc\). Then \(\kappa(\xc, x_s) = f(\xb) - f_{ni}(\xb)\) and we quantify the importance of \(\kappa\) as \(\mathbb{E}_{X_c, X_s} \left [ |\kappa(X_c, X_s)| \right ] \approx \sqrt{\frac{1}{N} \sum_i \kappa^2(\xc, x_s)} \).

    \subsection{Idea 2}
\end{document}

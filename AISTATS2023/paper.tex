\documentclass[twoside]{article}

\usepackage{aistats2023}
% If your paper is accepted, change the options for the package
% aistats2023 as follows:
%
%\usepackage[accepted]{aistats2023}
%
% This option will print headings for the title of your paper and
% headings for the authors names, plus a copyright note at the end of
% the first column of the first page.

% If you set papersize explicitly, activate the following three lines:
%\special{papersize = 8.5in, 11in}
%\setlength{\pdfpageheight}{11in}
%\setlength{\pdfpagewidth}{8.5in}

% If you use natbib package, activate the following three lines:
\usepackage[round]{natbib}
\renewcommand{\bibname}{References}
\renewcommand{\bibsection}{\subsubsection*{\bibname}}
\bibliographystyle{plainnat}
\usepackage{amsmath}
\usepackage{amssymb}

% If you use BibTeX in apalike style, activate the following line:
%\bibliographystyle{apalike}


\newcommand{\dfdx}{\frac{\partial f}{\partial x_s}}
\newcommand{\xc}{\mathbf{x_c}}
\newcommand{\DY}{\mathbf{\Delta Y}}
\newcommand{\xb}{\mathbf{x}}
\newcommand{\Xcb}{\mathbf{X}_c}
\newcommand{\Xb}{\mathcal{X}}

\begin{document}

% If your paper is accepted and the title of your paper is very long,
% the style will print as headings an error message. Use the following
% command to supply a shorter title of your paper so that it can be
% used as headings.
%
%\runningtitle{I use this title instead because the last one was very long}

% If your paper is accepted and the number of authors is large, the
% style will print as headings an error message. Use the following
% command to supply a shorter version of the authors names so that
% they can be used as headings (for example, use only the surnames)
%
%\runningauthor{Surname 1, Surname 2, Surname 3, ...., Surname n}

\twocolumn[

\aistatstitle{Instructions for Paper Submissions to AISTATS 2023}

\aistatsauthor{ Author 1 \And Author 2 \And  Author 3 }

\aistatsaddress{ Institution 1 \And  Institution 2 \And Institution 3 } ]

\begin{abstract}
  The Abstract paragraph should be indented 0.25 inch (1.5 picas) on
  both left and right-hand margins. Use 10~point type, with a vertical
  spacing of 11~points. The \textbf{Abstract} heading must be centered,
  bold, and in point size 12. Two line spaces precede the
  Abstract. The Abstract must be limited to one paragraph.
\end{abstract}


\section{INTRODUCTION}

Recently, ML has flourished in critical domains, such as healthcare
and finance. In these areas, we need ML models that predict accurately
but also with the ability to explain their predictions. Therefore,
Explainable AI (XAI) is a rapidly growing field due to the interest in
interpreting black box machine learning (ML) models. XAI literature
distinguishes between local and global interpretation
methods~\citep{Molnar2020interpretable}. Local methods explain a
specific prediction, whereas global methods explain the entire model
behavior. Global methods provide a universal explanation, summarizing
the numerous local explanations into a single interpretable outcome
(number or plot). For example, if a user wants to know which features
are significant (feature importance) or whether a particular feature
has a positive or negative effect on the output (feature effect), they
should opt for a global explainability technique. Aggregating the
individual explanations for producing a global one comes at a cost. In
cases where feature interactions are strong, the global explanation
may obfuscate heterogeneous effects~\citep{Herbinger2022repid} that
exist under the hood, a phenomenon called aggregation
bias~\citep{mehrabi2021survey}.

Feature effect forms a fundamental category of global explainability
methods, isolating a single feature's average impact on the
output. Feature effect methods suffer from aggregation bias because
the rationale behind the average effect might be unclear. For example,
a feature with zero average effect may indicate that the feature has
no effect on the output or, contrarily, it has a highly positive
effect in some cases and a highly negative one in others.

There are two widely-used feature effect methods; Partial Dependence
Plots (PDPlots)\citep{friedman2001greedy} and Aggregated Local Effects
(ALE)\citep{apley2020visualizing}. PDPlots have been criticized for
producing erroneous feature effect plots when the input features are
correlated due to marginalizing over out-of-distribution synthetic
instances. Therefore, ALE has been established as the state-of-the-art
feature effect method since it can isolate feature effects in
situations where input features are highly correlated.

However, ALE faces two crucial drawbacks. First, it does not provide a
way to inform the user about potential heterogeneous effects that are
hidden behind the average effect. In contrast, in the case of PDPlots,
the heterogeneous effects can be spotted by exploring the Individual
Conditional Expectations (ICE)\citep{goldstein2015peeking}. Second,
ALE requires an additional step, where the axis of the feature of
interest is split in \(K\) fixed-size non-overlapping intervals, where
\(K\) is a hyperparameter provided by the user. This splitting is done
blindly, which can lead to inconsistent explanations.

In this paper, we extend ALE with a probabilistic component for
measuring the uncertainty of the global explanation. The uncertainty
of the global explanation expresses how certain we are that the global
(expected) explanation is valid if applied to an instance drawn at
random and informs the user about the level of heterogeneous effects
hidden behind the expected explanation. Our method completes ALE, as
ICE plots complement PDPlots, for revealing the heterogeneous effects.

Our method also automates the step of axis splitting into
non-overlapping intervals. We, firstly, transform the bin splitting
step into an unsupervised clustering problem and, second, find the
optimal bin splitting for a robust estimation of (a) the global
(expected) effect and (b) the uncertainty of the explanation from the
limited samples of the training set. We formally prove that the
objective of the clustering problem has as lower-bound the aggregated
uncertainty of the global explanation. Our method works out of the box
without requiring any input from the user.

\paragraph{Contributions.} The contributions of this paper are the following:

\begin{itemize}
\item We introduce Uncertainty DALE (UDALE), an extension of DALE that
  quantifies the uncertainty of the global explanation, i.e.~the level
  of heterogeneous effects hidden behind the global explanation.
\item
  We provide an algorithm that automatically computes the optimal bin
  splitting for robustly estimating the explanatory quantities, i.e.,
  the global effect and the uncertainty.
\item We formally prove that our method finds the optimal grouping of
  samples, minimizing the added uncertainty over the unavoidable
  heterogeneity that is the lower-bound of the objective.
\item We provide empirical evaluation of the method in artificial and
  real datasets.
\end{itemize}


The implementation of our method and the code for reproducing all the
experiments is provided in the submission and will become publicly
available upon acceptance.


\section{BACKGROUND AND RELATED WORK}

\paragraph{Notation.} We refer to random variables (rv) using
uppercase \( X \), whereas to simple variables with plain lowercase
\( x \). Bold denotes a vector; \( \xb \) for simple variables or
\(\mathbf{X}\) for rvs. Often, we partition the input vector
\(\xb \in \mathbb{R}^D\) to the feature of interest
\(x_s \in \mathbb{R} \) and the rest of the features
\(\xc \in \mathbb{R}^{D-1}\). For convenience we denote it as
\((x_s, \mathbf{x}_c)\), but we clarify that it corresponds to the
vector \((x_1, \cdots , x_s, \cdots, x_D)\). Equivalently, we denote
the corresponding rv as \(X = (X_s, \mathbf{X}_c)\). The black-box
function is \(f : \mathbb{R}^D \rightarrow \mathbb{R}\) and the
feature effect of the \(s\)-th feature is
\(f^{\mathtt{<method>}}(x_s)\), where \(\mathtt{<method>}\) is the
name of the feature effect method.\footnote{An extensive list of all
  symbols used in the paper is provided in the helping material.}

\paragraph{Feature Effect Methods.} There are three well-known feature
effect methods: PDPlots, MPlots and ALE. PDPlots formulate the feature effect of the \(s\)-th
attribute as an expectation over the marginal distribution
\(\mathbf{X}_c\), i.e.,
\(f^{\mathtt{PDP}}(x_s) =
\mathbb{E}_{\mathbf{X}_c}[f(x_s,\mathbf{X}_c)]\). MPlots formulate it
as an expectation over the conditional \(\mathbf{X}_c|X_s\), i.e.,
\(f^{\mathtt{MP}}(x_s) = \mathbb{E}_{\mathbf{X}_c|X_s = x_s}[f(x_s,
\mathbf{X}_c)]\). ALE computes the global effect at \(x_s\) as an
accumulation (integration) of the expected value of the local
effects:
\begin{equation}
  \label{eq:ALE_accumulated_mean}
  f^{\mathtt{ALE}}(x_s) = \int_{z_{s,min}}^{x_s} \mathbb{E}_{\Xcb|X_s=z}\left[\frac{\partial f(z, \Xcb)}{\partial z}\right] \partial z
\end{equation}
%
ALE has specific advantages which gain particular value in cases of
corelated input features. In this cases, PDPlots integrate over
unrealistic instances, due to the use of the marginal distribution
\( p(\mathbf{X}_c) \), and MPlots compute aggregated effects, i.e.,
impute the combined effect of sets of features to a single
feature. ALE manages to resolve both issues, and is therefore the only
trustable method in cases of correlated features.

\paragraph{Quantify the Heterogeneous Effects.}

Feature effect methods answer the question \textit{what happens
  (effect) to the output, if I increase/decrease the value of a
  specific feature}. Having answered the question above, it comes
naturally to also wonder \textit{how certain we are about the change
  on the output}. For this reason, a lot of interest is given lately
for quantifying the level of uncertainty, along with the expected
effect. The level of uncertainty is mostly quantified by measuring the
existence of heterogeneous effects, i.e. whether there are local
explanations that deviate from the expected global effect. ICE and
d-ICE plots provide a visual understanding of the heterogeneous
effects on top of PDPs. Another approach targets on grouping the
heterogeneous effects, e.g., allocating ICE plots in homogeneous
clusters, by dividing the input space. Some other approaches, like
H-Statistic, Greenwel, move a step behind and try to quantify the
level of interaction between the input features, a possible cause of
heterogeneous effects. In this case, the interpretation is indirect,
since a strong interaction index indicates the possibility of
heterogeneous effects. The aforementioned approaches face two
pathogenies; They either do not quantify the uncertainty of the
feature effect directly or they are based on PDPs, and, therefore,
they are subject to the failure modes of PDPs in cases of correlated
features. To the best of our knowledge, no method so far targets on
quantify the uncertainty of the feature effect as it is modelled by
ALE.

\paragraph{Cluster Instances with homogeneous effects.}

In real ML scenarios, the expected feature effect and the uncertainty
are estimated from the limited instances of the training set. ALE
approximation requires an additional step, where the axis of the
\(s\)-th feature is split into a sequence of non-overlaping bins and a
single effect (expectation and uncertainty) is computed for from the
population of instances that lie inside each
bin. \citep{apley2020visualizing} proposed estimating the local
effects in each bin by evaluating the black box-funtion at the bin
limits:

\begin{equation}
  \label{eq:ALE_accumulated_mean_est}
  \hat{f}^{\mathtt{ALE}}(x_s) = \sum_{k=1}^{k_x} \frac{1}{|\mathcal{S}_k|} \sum_{i:\mathbf{x}^i \in
    \mathcal{S}_k} \left [ f(z_{k}, \xc^i) - f(z_{k-1}, \xc^i)) \right ]
\end{equation}

In contrast, (cite) proposed the Differential ALE (DALE) estimation
for quantifying the local effects on the training-set instances,
instead of the bin limits:

\begin{equation}
  \label{eq:DALE_accumulated_mean_est}
  \hat{f}^{\mathtt{ALE}}_{\mu}(x_s) = \Delta x \sum_{k=1}^{k_x} \frac{1}{|\mathcal{S}_k|} \sum_{i:\mathbf{x}^i \in
    \mathcal{S}_k} \frac{\partial f}{\partial x_s}(\mathbf{x}^i)
\end{equation}
%
Their method has the advantages of remaining on-distribution even when
bins become wider and, most importantly, it allows the recomputation
of the accumulated effect with different bin-splitting with near-zero
computational overhead. However, none of the approximations above
deals with the crucial problem of the optimal bin-spliting. They split
the axis in a blind-way, partition in \(K\) equally-sized which can
lead to erroneous approximations.

Instead, we propose treating the axis-spliting step as an unsupervised
clustering problem. The objective of the clustering problem should
fulfill in the best way to contradictory ojectives. First, secure
robust estimations of the expected effect and the uncertainty inside
each bin given the limited instances of the training set and, second,
create bins with as homogeneous local effects as possible, for not
losing fine-grain resolution feature effects due to wide bins.

\section{THE ... METHOD}

\subsection{ALE with Uncertainty Quantification}

ALE defines the local effect of the \(s\)-th feature on \(f(\cdot)\)
at point \((x_s, \xc)\) as \(\dfdx (x_s, \xc)\). All the local
explanations at \(x_s\) are, then, weigthed by the conditional
distribution \(p(\xc|x_s)\) and are averaged, to produce the
summarized effect at \(x_s\):

\begin{equation}
  \label{eq:ALE_mean}
  \mu(x_s) = \mathbb{E}_{\Xcb|x_s}\left [\dfdx (x_s, \Xcb)\right ]
\end{equation}
\noindent
As described at the Introduction, limiting the explanation to the
exepected value level does not shed light to possible heterogeneous
effects behind the averaged explanation. Therefore, we model the
uncertainty of the local effects at \(\mathcal{H}(x_s)\) as the
variance of the local explanations:

\begin{equation}
  \label{eq:ALE_var}
  \mathcal{H}(x_s) := \sigma^2(x_s) = \mathrm{Var}_{\Xcb|x_s}\left [\dfdx (x_s, \Xcb) \right ]
\end{equation}
\noindent
The uncertainty of the explanation emerges from the natural
characteristics of the experiment, i.e.,~the data generating
distribution and the properties of the black-box function. In Section
(TODO), we propose appropriate visualizations for easier
interpretation of Eq.~(\ref{eq:ALE_var}). In ALE, the feature effect
at \(x_s\) is the accumulation of the averaged local effects from
\(x_{min}\) until \(x_s\), as show in
Eq.~(\ref{eq:ALE_accumulated_mean}). Equivalently, we define the
accumulated uncertainty (variance) until the point \(x_s\), as the
integral of the variances of local effects:

\begin{equation}
  \label{eq:ALE_accumulated_var}
  f^{\mathtt{ALE}}_{\sigma^2}(x_s) = \int_{z_{s, min}}^{x_s} \sigma^2(z) \partial z
\end{equation}
\noindent

The accumulated uncertainty is not a directly interpretable
quantity. It only helps us define a sensible objective for the
interval spliting step, as we discuss in Section TODO: adde ref.

\subsection{Uncertainty Quantification and Estimation at an Interval}

In real scenarios, we have ignorance about the data-generating
distribution \(p(x_s, \mathbf{x}_c)\) and all estimations are based on
the limited instances of the training set. Estimating
Eqs.~\eqref{eq:ALE_mean},~\eqref{eq:ALE_var} at the granularity of a
point \(x_s\) is impossible, because the probability of observing a
sample inside the interval \([x_s - h, x_s + h]\) tends to zero, when
\(h \to 0\). Therefore, we are obliged to split the axis of \(x_s\)
into a sequence of non-overlaping intervals (bins) and estimate the
mean and the variance from the samples that lie inside each bin. The
mean effect at an interval \([z_1, z_2)\) is defined as the mean of
the expected effects:

\begin{equation}
  \label{eq:mu_bin}
  \mu(z_1, z_2) = \frac{1}{z_2 - z_1} \int_{z_1}^{z_2}
  \mathbb{E}_{\xc|x_s=z}\left [\frac{\partial f}{\partial x_s} \right ] \partial z
\end{equation}

\noindent
Accordingly, the accumulated variance at an interval \([z_1, z_2)\)
is defined as:

\begin{equation}
  \label{eq:var_bin}
  \sigma^2(z_1, z_2) = \int_{z_1}^{z_2}
  \mathbb{E}_{\xc|x_s=z} \left [ (\frac{\partial
      f}{\partial x_s} - \mu(z_1, z_2) )^2 \right] \partial z
\end{equation}

Eqs.~\eqref{eq:mu_bin},~\eqref{eq:var_bin} can be directly estimated
from the set \(\mathcal{S}\) of the dataset instances with the
\(s\)-th feature lying inside the interval, i.e.,
\( \mathcal{S}= \{ \mathbf{x}^i : z_1 \leq x^i_s < z_2 \} \). The mean
effect at the interval, Eq.~(\ref{eq:mu_bin}) is approximated by:

\begin{equation}
  \label{eq:mean_estimation}
  \hat{\mu}(z_1, z_2) = \frac{1}{|\mathcal{S}|} \sum_{i:\mathbf{x}^i \in
    \mathcal{S}} \left [ \dfdx(\mathbf{x}^i) \right ]
\end{equation}

and the accumulated variance, Eq.~(\ref{eq:var_bin}) can be
approximated by

\begin{equation}
  \label{eq:variance_estimation}
  \hat{\sigma}^2(z_1, z_2) = \frac{z_2 - z_1}{|\mathcal{S}|} \sum_{i:\mathbf{x}^i \in
    \mathcal{S}} \left ( \dfdx(\mathbf{x}^i) - \hat{\mu}(z_1, z_2) \right )^2
\end{equation}


The approximation is unbiased only if the points are uniformly
distributed in \([z_1, z_2]\). (TODOs: Check what happens
otherwise).

\subsection{Bin Spliting as a Clustering Problem}

ALE, Eq.(\ref{eq:ALE_accumulated_mean}), is estimated by spliting the
axis \(x_s\) into a sequence of non-overlaping bins (TODO: add some
discussion for ALE, DALE):

\begin{equation}
  \label{eq:ALE_accumulated_mean_est}
  \hat{f}^{\mathtt{ALE}}_{\mu}(x_s) = \Delta x \sum_{k=1}^{k_x} \hat{\mu}(z_{k-1}, z_k)
\end{equation}
and 

\begin{equation}
  \label{eq:ALE_accumulated_var_est}
  \hat{f}^{\mathtt{ALE}}_{\sigma^2}(x_s) = \sum_{k=1}^{k_x} \hat{\sigma}^2(z_{k-1}, z_k)
\end{equation}

We denote as \(k_x\) the index of the bin that \(x_s\) belongs to,
i.e. \(k_x: z_{k_x-1} \leq x_s < z_{k_x} \) and \(\mathcal{S}_k\) is
the set of training instance that lie in the \(k\)-th bin, i.e.
\( \mathcal{S}_k = \{ \xb^i : z_{k-1} \leq x^i_s < z_{k} \} \). Both
methods face the limitation that the partitioning into non-overlaping
intervals is done blindly. The user pass the total number of bins
\(K\) as a hyperparameter, the bins are defined with equal-size
spliting, and the training instances are allocated accordingly. This
approach is vulnerable to non-robust estimations. The mean effect is
often poorly approximated from a very small number of samples and the
mean effect of empty bins is interpolated from their
neighboors. Furthermore, in our case, we need sufficient sample
populations for estimating the variance of the approximation, apart
from the mean effect.

\subsubsection{Methodology}

For overcoming this limitations, we reformulate the partitioning as a
clustering of the training instances into a sequence variable-size
intervals. The objective of the clustering problem is inspired 


ALE requires splittingthe estimation of the 

In this section, we introduce a framework 

\paragraph{Theorem 1.} If we define the residual \(\rho(z)\) as the
difference between the expected effect at \(x_s\) and the mean
expected effect at the interval, i.e
\(\rho(z) = \mu(z) - \mu(z_1, z_2)\), then, the accumulated variance
at an interval \([z_1, z_2)\) is the accumulation of the all variances
plus the accumulation of squared residuals inside the interval:

\begin{equation}
 \sigma^2(z_1, z_2) = \int_{z_1}^{z_2} \sigma^2(z) + \rho^2(z) \partial z
\end{equation}
%
The proof is at the Appendix. Theorem 1 decouples the accumulated
variance at an interval, the only quantity we can estimate, into two
terms. The first term \(\int_{z_1}^{z_2} \sigma^2(z) \partial z\),
quantifies the uncertainty due to the natural characteristics of the
experiment and the second term adds extra uncertainty due to the
limited resolution.



\paragraph{Uncertainty of the global effect.}

Eq.~\eqref{eq:variance_estimation} gives an approximation of the
uncertainty of the bin effect.
The uncertainty of the global effect is
simply the sum of the uncertainties in the bin effects.

\paragraph{Minimizing the uncertainty}

Solving the problem of finding (a) the optimal number of bins \(K\) and (b) the optimal bin limits for each bin \([z_{k-1}, z_k] \forall k\) to minimize:

\begin{equation}
  \label{eq:1}
  \mathcal{L} = \sum_{k=0}^K \hat{\sigma}_k(z_{k-1}, z_k)
\end{equation}
%
The constraints are that all bins must include more than \(\tau\)
points, i.e., \(|\mathcal{S}_k| \geq \tau\).

\noindent
TODOS. Show theoretically that \(\mathcal{L} \geq \int_{x_{s, \min}}^{x_{s, \max}}\sigma^2(x_s) \partial x_s\)


% Furthermore, we automate the step of splitting the axis into non-overlapping intervals. The need for non-overlapping bins emerges from the ignorance of the data-generating distribution, enforcing all estimations to be based on the limited instances of the training set. Therefore, there is an implicit trade-off behind the formation of bins. Each bin must include enough instances for a robust estimation of the bin feature effect (expected value), and the uncertainty of the explanation (variance). On the other hand, each bin should include points with similar local effects. Therefore, we transform the bin splitting step into an unsupervised clustering problem, encoding the trade-off mentioned above in the objective function. We formally show that the objective of the clustering problem has lower-bound the (unavoidable) heterogeneity, i.e., the aggregated uncertainty of the global explanation. Therefore, we aim to find the optimal grouping of samples that adds the slightest uncertainty over the unavoidable heterogeneity. We finally solve the minimization problem by finding the global optimum using dynamic programming. Our method works out of the box without requiring any input by the user. We provide a theoretical and empirical evaluation of our method.


\subsection{Visualization of ALE with Uncertainty}

\section{SYNTHETIC EXAMPLES}

\section{REAL-WORLD EXAMPLES}


\subsubsection*{Acknowledgements}
All acknowledgments go at the end of the paper, including thanks to reviewers who gave useful comments, to colleagues who contributed to the ideas, and to funding agencies and corporate sponsors that provided financial support. 
To preserve the anonymity, please include acknowledgments \emph{only} in the camera-ready papers.

\bibliography{biblio}


\section*{Appendix}

\subsection{Proof for variance of the bin}



\end{document}

\documentclass[12pt]{article}
\usepackage{amsfonts}
\usepackage{amsmath}
\usepackage{bm}

\title{Research Ideas}

\newcommand{\xb}{\mathbf{x}}
\newcommand{\xc}{\mathbf{x_c}}
\newcommand{\fxc}{f^{(\xc)}}
\newcommand{\fxs}{f^{(x_s)}}

\author{Vasilis Gkolemis}
\begin{document}
    \maketitle
    \section{Evaluation framework for interaction methods}

    \subsection{Idea 1}

    We may split every $f: \mathbb{R}^D \rightarrow \mathbb{R}$ into
    a model without interaction between $\xc$ and $x_s$,
    i.e., $f_{ni}(\xb) = \fxs(x_s) +  \fxc(\xc)$,
    and the interaction term $\kappa(\xc, x_s)$:

    \[
      f(\xb) = \underbrace{\fxs(x_s) +  \fxc(\xc)}_{f_{ni}(\xb)} + \kappa(\xc, x_s)
    \]

    A simple approach is defining \(f\) to be a Neural Network and \(f_{ni}\) a Neural Additive Model without interaction
    between \(x_s\) and \(\xc\). Then \(\kappa(\xc, x_s) = f(\xb) - f_{ni}(\xb)\) and we quantify the importance of \(\kappa\) as \(\mathbb{E}_{X_c, X_s} \left [ |\kappa(X_c, X_s)| \right ] \approx \sqrt{\frac{1}{N} \sum_i \kappa^2(\xc, x_s)} \).

    \subsection{Idea 2}
\end{document}
